\documentclass[article,nojss,shortnames]{jss}

\usepackage{thumbpdf,lmodern}
\usepackage{framed}
\usepackage{algorithmic}
\usepackage{amsmath}
\usepackage[]{algorithm2e}

\newcommand{\note}[1]{\textcolor{red}{\textbf{Note:} #1}}


% Draft
%\hypersetup{draft}
%\shortcites{rasmussen2012}

\author{Reto Stauffer\\Universit\"at Innsbruck
\And Matthias Dusch\\Universit\"at Innsbruck
\AND Fabien Maussion\\Universit\"at Innsbruck
\And Georg J. Mayr\\Universit\"at Innsbruck}

\Plainauthor{Reto Stauffer, Matthias Dusch, Fabien Maussion, Georg J. Mayr}

\title{phoeton: Software for Objective Probabilistic Foehn Classification}
\Plaintitle{phoeton: Objective Probabilistic Foehn Classification}
\Shorttitle{phoeton: Objective Probabilistic Foehn Classification}

\Abstract{
    This is the abstract here \dots
}

\Keywords{meteorology, foehn, objective classification, flexible mixture model}
\Plainkeywords{meteorology, foehn, objective classification, flexible mixture model}

\Address{
  Reto Stauffer\\
  Department of Statistics\\
  Faculty of Economics and Statistics\\
  Universit\"at Innsbruck\\
  Universit\"atsstr.~15\\
  6020 Innsbruck, Austria\\
  E-mail: \email{Reto.Stauffer@uibk.ac.at}\\
  URL: \url{https://retostauffer.org}\\

  Matthias Dusch, Fabien Maussion, Georg J. Mayr\\
  Department of Atmospheric and Cryospheric Sciences\\
  Faculty of Geo- and Atmospheric Sciences\\
  Universit\"at Innsbruck\\
  Innrain~52f\\
  6020 Innsbruck, Austria\\
  E-mail: \email{Georg.Mayr@uibk.ac.at},
          \email{Matthias.Dusch@uibk.ac.at},
          \email{Fabien.Maussion@uibk.ac.at}
}


\begin{document}

% -------------------------------------------------------------------
% SECTION: INTRODUCTION
% -------------------------------------------------------------------
\section{Introduction}

Introduction why this is so great. Georg might like to contribute
here. Just to leave/include some references:

\begin{itemize}
    \item finite mixture model \citep{leisch2004,gruen2008}
    \item automatic probabilistic foehn classification \cite{plavcan2014}
    \item IWLS/IRLS \citep{mccullagh1989}
\end{itemize}



% -------------------------------------------------------------------
% SECTION: METHODS
% -------------------------------------------------------------------
\section{Methods}

\textit{Note by Reto:}
The original publication (\code{FlexMix}, \citealt{gruen2008})
provides general framework for flexible mixture models for multiple
clusters. For this application we will concentrate on the special
case with only two distinct clusters ($K = 2$).

\subsection{Finite mixture models}

The finite mixture model in its general form is given by

\begin{equation}
    h(\mathit{y} | \mathit{x}, \mathit{\omega}, \mathit{\alpha}, \mathit{\theta}) =
        \sum_{k=1}^K \pi_k(\mathit{\alpha}, \mathbf{\omega}) \cdot f_k(\mathit{y} | \mathit{x}, \mathit{\theta}_k)
        ~~\text{for}~~k=1,\dots,K
    \label{eqn:flexmix-density-general}
\end{equation}

where $\mathit{\theta}$ and $\mathit{\alpha}$ denote the parameters for the
mixture density $h()$. $\mathit{y}$ denotes the response, $\mathit{x}$ the
predictor variable (one single variable) and $\mathbf{\omega}$ the concomitant
variables (multiple variables allowed) for the probability model.

For the special case of a finite mixture model with $K=2$ Gaussian clusters
the specification as shown in Equation~\ref{eqn:flexmix-density-general} simplifies to

\begin{equation}
    h(\mathit{y} | \mathit{x}, \mathit{\omega}, \mathit{\alpha}, \mathit{\theta}) =
        \underbrace{
            (1 - \mathit{\pi}(\mathbf{\omega}, \mathit{\alpha})) \cdot
            \phi(\mathit{y} | \mathit{\theta}_1, \mathit{x})
        }_{\text{cluster 1}}
        \cdot
        \underbrace{
            \mathit{\pi}(\mathbf{\omega}, \mathit{\alpha}) \cdot \phi(\mathit{\theta}_2|\mathit{x})
        }_{\text{cluster 2}}
    \label{eqn:flexmix-density-gaussian}
\end{equation}

where $\phi()$ is the probability function of the Gaussian distribution.
The concomitant model for $\pi()$ reduces to a binomial problem as the probability
for a specific observation $y_i$ to be observed in the first cluster is the
counter probability of $y_i$ being observed in the second cluster.
Thus, for $\pi()$ any type of binomial model can be used which satisfies
that $\pi \in [0,1] \forall i=1,\dots,N$.

One frequently used model for a binary response is the binomial logistic regression
model or binomial logit model (BLM), however, any appropriate binomial probability
model could be used for $\pi$.
The BLM is defined as follows:

\begin{equation}
    \log\Big(\frac{\pi}{1 - \pi}\Big) =
        \omega^\top \alpha = \alpha_0 + \omega_1 \alpha_1 + \dots + \omega_P \alpha_P,
\end{equation}

where $\mathit{\pi}$ is the probability that an observation belongs to the
second cluster (Eqn.~\ref{eqn:flexmix-density-gaussian}).
The probabilities are modelled using multilinear regression
$\mathit{\omega}^\top \mathit{\alpha}$ with $p=1,\dots,P$ concomitant
variables $\mathit{\omega}$ times the corresponding parameters or regression
coefficients $\mathit{\alpha}$.

As the BLM is a generalized linear model, the parameters can be estimated using
an iterative (re-)weighted least squares (IWLS) solver.
The IWLS procedure to estimate the parameters $\mathit{\alpha}$ of the BLM
is shown in Algorithm~\ref{alg:iwls}.

\begin{algorithm}
    \caption{Iterative (re-)weighted least squares (IWLS) solver for the binomial logit model (BLM).}
    \label{alg:iwls}

    \textbf{Initialization ($j = 0$):}
    \begin{enumerate}
        \item Initialize $\mathit{\alpha}^{0} = 0$.
        \item Compute initial latent response $\mathit{\eta}^{(0)} = \mathbf{\omega}^\top\mathit{\alpha}^{(0)}$.
        \item Compute initial probabilities $\pi^{(0)} = \frac{\exp(\eta^{(0)})}{1 + \exp(\eta^{(0)})}$. 
        \item Evaluate initial log-likelihood:
            \begin{equation*}
                \ell^{(0)} = %(\mathit{\alpha} | \mathbf{\omega})^{(0)} =
                \sum_{i=1}^{N} \Big(\log(1 - \pi_i^{(0)})^{(y_i = 0)} + \log(\pi_i)^{(y_i=1)}\Big)
            \end{equation*}

    \end{enumerate}
    \textbf{Iterative parameter estimation ($j = 1, \dots, J$):}
    \begin{enumerate}
        \item Compute new inverse square root weights:
            \begin{equation*}
                \mathit{W}^{(j)} = \Big(\big(\frac{d\eta}{d\pi}\big)^{(j-1)}\Big)^{-\frac{1}{2}} =
                    \big(\mathit{\pi}^{(j-1)} (1 - \mathit{\pi}^{(j-1)})\big)^\frac{1}{2}
            \end{equation*}
        %%%\item Compute new latent response:
        %%%    \begin{equation*}
        %%%        \mathit{z}^{(j+1)} = \mathit{\eta}^{(j)} + (\mathit{y} - \mathit{\pi}^{(j)})
        %%%            \Big(\frac{d \mathit{\eta}}{d \mathit{\pi}}\Big)^{(j)} =
        %%%            \mathit{\eta}^{(j)} + \frac{\mathit{y} - \mathit{\pi}^{(j)}}{\mathit{\pi}^{(j)}(1 - \mathit{\pi}^{(j)})}
        %%%    \end{equation*}
        \item Update parameters:
            \begin{equation*}
                \mathit{\alpha}^{(j)} =
                    \big((\mathbf{\omega} \mathit{W}^{(j)})^\top
                    (\mathbf{\omega} \mathit{W}^{(j)})\big)^{-1}
                    (\mathbf{\omega} \mathit{W}^{(j)})^\top
                    \mathit{W}^{(j)} \mathit{z}^{(j)}
            \end{equation*}

            with $\mathit{z}^{(j)} = \mathit{\eta}^{(j-1)} + (\mathit{y} - \mathit{\pi}^{(j-1)}) \Big(\frac{d\pi}{d\eta}\Big)^{(j-1)}$ this yields

            \begin{equation*}
                \mathit{\alpha}^{(j)} =
                    \big((\mathbf{\omega} \mathit{W}^{(j)})^\top
                    (\mathbf{\omega} \mathit{W}^{(j)})\big)^{-1}
                    (\mathbf{\omega} \mathit{W}^{(j)})^\top
                    \Big(\mathit{\eta}^{(j-1)} \mathit{W}^{(j)} + \frac{\mathit{y} - \mathit{\pi}^{(j-1)}}{W^{(j)}}\Big)
            \end{equation*}

        \item Update response:
            \begin{equation*}
                \mathit{\eta}^{(j)} = \mathbf{\omega}\mathit{\alpha}^{(j)}
                ~~\text{and}~~
                \mathit{\pi}^{(j)} = \frac{\exp(\mathit{\eta}^{(j)})}{1 + \exp(\mathit{\eta}^{(j)})}
            \end{equation*}

        \item Evaluate log-likelihood sum:
            \begin{equation*}
                \ell^{(j)} = %(\mathit{\alpha}^{(j+1)} | \mathbf{\omega}) =
                \sum_{i=1}^{N} \Big(\log(1 - \pi_i^{(j)})^{(y_i = 0)} + \log(\pi_i^{(j)})^{(y_i=1)}\Big)
            \end{equation*}

        \item Repeat until convergence, e.g., until the log-likelihood improvement
            $\ell^{(j)} - \ell^{(j-1)}$ falls below a certain threshold or $j$ reaches
            $J$, the maximum number of iterations.

    \end{enumerate}
\end{algorithm}



\subsection{Model estimation}

The parameters $\mathit{\theta}$ and $\mathit{\alpha}$ of the finite mixture model
can be estimated using a likelihood based EM algorithm (\note{citation?}).

For the special case of a two-cluster Gaussian finite mixture model the estimation
can be performed using iterative (re-)weighted empirical moments for the parameters
of the Gaussian densities ($\mathit{\theta}$) and IWLS to estimate the parameters
of the binomial logit model ($\mathit{\theta}$).
\note{This is true for $K>2$ as well, only the probability model would have to be adjusted.}

It is assumed that an unobservable latent variable $z_i \in [0,1]$ exists
for each observation $i=1, \dots N$ which defines the membership of the observation,
the probability that observation $n$ comes from the second cluster.
As the membership is unknown it has to be initially guessed:

\begin{equation}
    \mathit{z}^{(0)} = \begin{cases}
        1 & \text{if}~x_n > \text{median}(x), \\
        0 & \text{else}.
    \end{cases}
    \label{eqn:init-zn}
\end{equation}
\note{This $z$ is basically the initial guess of the probability!}

In addition, initial values for the parameters of the two Gaussian distributions
for cluster one and cluster two have to be estimated which are used as starting
values for the EM procedure. The parameters $\mathit{\theta}$ are thus initialized
as follows:

\begin{equation}
    \begin{split}
        \mathit{\theta}_1^{(0)} = (\mu_1^{(0)}, \sigma_1^{(0)}) = \Big(q_{0.25}(x), \text{sd}(x)\Big) \\
        \mathit{\theta}_2^{(0)} = (\mu_2^{(0)}, \sigma_2^{(0)}) = \Big(q_{0.75}(x), \text{sd}(x)\Big)
    \end{split}
    \label{eqn:init-theta}
\end{equation}

The first ($q_{0.25}()$) and the third ($q_{0.75}()$) quartile of the predictor 
variable $\mathit{x}$ are used as inital guess for the location parameters
$\hat{\mu}_1$ and $\hat{\mu}_2$ while the standard deviation of $\mathit{x}$ is used as initial
guess for the scale parameter $\hat{\sigma}_1$ and $\hat{\sigma}_2$.
The parameters $\mathit{\alpha}$ of the binomial logit model are simply initialized
with $0$ (\note{find a way for better initial values}).



The EM algorithm can be written as follows:

\begin{algorithm}
    \caption{Parameter estimation for the two-cluster Gaussian finite mixture model
    using iterative (re-)weighted optimization. $\Phi()$ is the cumulative distribution
    function of the standard Gaussian distribution.}
    
    \textbf{Initialization ($j = 0$):}
    \begin{enumerate}
        \item Initialize $\mathit{z}$ and $\mathit{\theta}$ as shown in
            Equation~\ref{eqn:init-zn}\,\&\,\ref{eqn:init-theta}.\label{item:a2:1}

        \item Estimate initial $\alpha^{(0)}$ parameters using $\mathit{z}$ from
            Step~\ref{item:a2:1} as $\mathit{y}$ and the IWLS solver from
            Algorithm~\ref{alg:iwls}.

        \item Compute response of the BLM with initial parameters:
            \begin{equation*}
                \mathit{\eta}^{(0)} = \mathbf{\omega}^\top \mathit{\alpha}^{(0)}
                ~~\text{and}~~
                \mathit{\pi}^{(0)} = \frac{\exp(\mathit{\eta}^{(0)})}{1 - \exp(\mathit{\eta}^{(0)})}
            \end{equation*}
    \end{enumerate}

    \textbf{EM iterations ($j = 1, \dots, J$):}
    \begin{enumerate}

        \item Compute finite mixture model densities for both clusters:
            \begin{equation*}
                    d_1^{(j)} = (1 - \mathit{\pi}^{(j)}) \Phi\Big(\frac{\mathit{y} - \mu_1^{(j-1)}}{\sigma_1^{(j-1)}}\Big),
                    ~~
                    d_2^{(j)} = \mathit{\pi}^{(j)} \Phi\Big(\frac{\mathit{y} - \mu_2^{(j-1)}}{\sigma_2^{(j-1)}}\Big)
            \end{equation*}

        \item Compute posterior probabilities:
            \begin{equation*}
                \tilde{\mathit{p}}^{(j)} = \frac{d_2^{(j)}}{d_1^{(j)} + d_2^{(j)}}
            \end{equation*}

        %%%\item Compute initial log-likelihood sum:
        %%%    \begin{equation*}
        %%%        \ell^{(j)} = \sum_{i=1}^N 
        %%%        \underbrace{
        %%%            (1 - \tilde{p}_i^{(j)}) \log(d_{1,i}^{(j)})
        %%%        }_{\text{cluster 1}}
        %%%        +
        %%%        \underbrace{
        %%%            \tilde{p}_i^{(j)} \log(d_{2,i}^{(j)})
        %%%        }_{\text{cluster 2}}
        %%%        +
        %%%        \underbrace{
        %%%            (1 - \tilde{p}_i^{(j)}) \log(1 - \pi_i^{(j)}) + \tilde{p}_i^{(j)} \log(\pi_i^{(j)})
        %%%        }_{\text{concomitant model}}
        %%%        \Big)
        %%%    \end{equation*}
            
        \item Update $\mathit{\theta}$ using weighted empirical moments:
            \begin{equation*}
                \begin{split}
                    \mu_1^{(j)} = \frac{\sum \mathit{x} (1 - \tilde{p}^{(j)})}{\sum (1 - \tilde{p}^{(j)})}
                    ~~\text{and}~~
                    \sigma_1^{(j)} = \Big( \frac{\sum (x - \mu_1^{(j)})^2 (1 - \tilde{p}^{(j)})}{\sum(1 - \tilde{p}^{(j)})} \Big)^\frac{1}{2} \\
                    \mu_2^{(j)} = \frac{\sum \mathit{x} \tilde{p}^{(j)}}{\sum \tilde{p}^{(j)}}
                    ~~\text{and}~~
                    \sigma_2^{(j)} = \Big( \frac{\sum (x - \mu_2^{(j)})^2 \tilde{p}^{(j)}}{\sum \tilde{p}^{(j)}} \Big)^\frac{1}{2} \\
                \end{split}
            \end{equation*}

        \item Update $\mathit{\alpha}$ parameters using the new probabilities $\pi^{(j)}$
            as $\mathit{y}$ and the IWLS solver from Algorithm~\ref{alg:iwls}.

        \item Update response of the concomitant model:
            \begin{equation*}
                \mathit{\eta}^{(j)} = \mathbf{\omega}\mathit{\alpha}^{(j)}
                ~~\text{and}~~
                \mathit{\pi}^{(j)} = \frac{\exp(\mathit{\eta}^{(j)})}{1 + \exp(\mathit{\eta}^{(j)})}
            \end{equation*}

        \item Evaluate log-likelihood given the new parameters $\mathit{\alpha}$ and $\mathit{\theta}$
            as well as the updated posterior probabilities $\mathit{\tilde{p}}$:
            \begin{equation*}
                \ell^{(j)} = \sum_{i=1}^N 
                \underbrace{
                    (1 - \tilde{p}_i^{(j)}) \log(d_{1,i}^{(j)})
                }_{\text{cluster 1}}
                +
                \underbrace{
                    \tilde{p}_i^{(j)} \log(d_{2,i}^{(j)})
                }_{\text{cluster 2}}
                +
                \underbrace{
                    (1 - \tilde{p}_i^{(j)}) \log(1 - \pi_i^{(j)}) + \tilde{p}_i^{(j)} \log(\pi_i^{(j)})
                }_{\text{concomitant model}}
                \Big)
            \end{equation*}

        \item Stop if $j = J$ and the maximum number of iterations is reached or
            (for $j > 1$) if the improvements in the log-likelihood $\ell^{(j)} - \ell^{(j-1)}$
            fall below a certain threshold.
    \end{enumerate}
\end{algorithm}


\begin{equation}
    z_{i1} = \begin{cases} 0 & \text{if} ~~ y_i < \text{median}(y) \\ 1 & \text{else}\end{cases}
\end{equation}


\begin{equation}
    z = \frac{\exp(\omega^\top \alpha)}{1 + \exp(\omega^\top \alpha)}
\end{equation}

The following lines prepare the prior information $z$, $\theta$, and $\alpha$:



% -------------------------------------------------------------------
% SECTION: SOFTWARE
% -------------------------------------------------------------------
\section{Software}

Explain software.


% -------------------------------------------------------------------
% SECTION: DATA
% -------------------------------------------------------------------
\section{Data}

Maybe just combine data and the case study.

% -------------------------------------------------------------------
% SECTION: RESULTS
% -------------------------------------------------------------------
\section{Case study/results}


% -------------------------------------------------------------------
% SECTION: Discussion and Outlook
% -------------------------------------------------------------------
\section{Discussion and outlook}


% -------------------------------------------------------------------
% Copernicus extra sections
% -------------------------------------------------------------------


\section*{Acknowledgments}

Acknowledgments dow here \dots


% -------------------------------------------------------------------
% Bibliography here
% -------------------------------------------------------------------
\newpage
\bibliography{references}

% -------------------------------------------------------------------
% Appendix section
% -------------------------------------------------------------------
\newpage

\begin{appendix}

\section{Appendix Section}

This is the appendix, if needed \dots

\end{appendix}

\end{document}

